\chapter{Development}\label{ch:development}
\section{Component List}
\subsection{Introduction}
To create the basis for development of this project two microphones and a structure to keep a constant distance between them were necessary.  

\subsection{Microphones characteristics}
Firstly a decision had to be made whether to use a dicetional or omnidirectional microphone. Since omnidirectional microphones provide less plosive and wind sounds, does not build up bass it was considered a good option. The fact that it provides equally good audio quality at every angle made it the best choice for the aim of this project.
\paragraph{Our requirements for the microphones were: \\}

\paragraph{• Affordable price\\}   
Since the university did not have any planned funding for this semester, it was agreed to aim for budget options. This way it would not be necessary to work towards an agreement with the University to receive funding - some members wanted to have microphones for themselves.
\paragraph{• Capability to capture the entire range of human hearing frequency\\}   
Since one of the great benefits of an omnidirectional microphone was it's ability to capture clear sound, it would not have been reasonable to purchase microphones that were not able to capture the full clarity of the input. Since the entire human voice frequency fits within our hearing frequency range, there is no need to have any extra requirements.
\paragraph{• Appropriate size factor\\}    
It would be beneficial and logical if the microphones used in the testing would have potential to be a part of the design prototype. Although due to limited budget there is a low chance of affording the microphones with an appropriate size parameters for a hearing aid or earphones, objective to apply a microphone that could be used in the further stages of development remains as one of lower priorities    
\paragraph{• They must to be identical\\}    
Having identical microphones assures that if there was a delay within recording timings, it would not be caused by hardware differences.
\paragraph{• Capability to connect multiple microphones to a single computer and record both at the same time\\}   
For the sake of simplicity regarding sampling and testing, it was agreed that it would be far easier to sample using microphones that can be connected though Universal Serial Bus port instead of Auxiliary one. This way troubleshooting and initiating microphones should be more clear. 

\subsection{Microphones}
As requirements for microphones were set, it was first attempted to find them at the University. As it was found out that University does not own any microphones that could be applied for the purpose of this project, it was agreed to purchase budget microphones that could come as close to the requirements raised for the project as possible. After looking though options, bearing in mind that a month-long delivery from Asia is not a viable option, it was decided to order microphones available within an acceptable delivery time \todo{reference}.\\

\subsection{Distancing}
To maintain a constant distance it was decided to attempt modeling a rod with two microphone holders at it's ends. The distance chosen between the microphones was agreed on by referring to a research article in this area: 15.2 centimeters \todo{reference}. 
\paragraph{}
Other parameters were found by following the measurements found in the datasheet\todo{reference}. These steps have led us towards the structure developement, which after a few iterations has turned into the structure that was used for sampling \todo{illustration missing (3D model or real one)}.
\begin{figure} [htp] 
  \centering
    \includegraphics[width=0.5\textwidth]{Illustrations/MicData}
    \caption{Microphones data sheet}
    \label{fig:MicData}
\end{figure}

\todo[inline,color=green]{\url{https://www.researchgate.net/publication/228749231_Analysis_of_the_Facial_Anthropometric_Data_of_Korean_Pilots_for_Oxygen_Mask_Design}}

\subsection{Measurements scenarios}

\subsection{Setup}
\section{Analog to digital conversion}



