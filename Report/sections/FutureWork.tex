\chapter{Future Work}\label{FutureWork}
\subsection{State of hearing aid}
\paragraph{}
Hearing aid technologies have never been improving as fast as they do now. Ever since major smartphone manufacturer companies started investing into wireless wearable technology research, price of PSAP (personal sound amplification products) has decreased while the amount of features has increased. Throughout recent years wireless earphones became predominant in the global market due to new generation of Bluetooth technology. This improvement even further reduces power consumption of wireless technology. The only three differences between hearing aids and PSAPs were: \\

•The battery life. Due to a far higher number of features and consistent audio stream PSAPs consume a much higher amount of power.\\

•Hardware design differences. PSAPs are not oriented around sound localization to inform the user about where the sound is coming from regarding natural sources. PSAPs are often not oriented to be invisible to others, they are more often purposely made to stand out and be recognized among its competitors. Fit customization is also often minimal on PSAPs.\\

•Regulation requirements to produce the hearing aid and license requirements to sell it.

\paragraph{} 
Most other differences lay in software and could be eliminated through a software update.
\paragraph{}
It is believed that legislative issues could be solved if manufacturers would put effort to reach for an agreement with legislators although it would require a lot of changes since current hearing aid selling process consists of far more than just taking the product off the shelf and swiping it through the register - it is normally performed at hearing clinics, hearing aid is thoroughly adjusted to fit the consumer's ear for long periods of time, warranty for these devices also is taken in a far more serious manner: it comes with included follow-up office visits, checks and cleaning procedures to maintain the highest level of performance. Some companies do express interest to merge the two markets. According to "The State of Hearing Healthcare 2017" by Lindsey Banks, "If Apple Air Pods or Samsung Gear IconX could add in hearing aid functions, that’s instant access to over half of the U.S. over night."  
\todo[inline,color=green]{\url{https://www.everydayhearing.com/hearing-loss/articles/state-of-hearing-healthcare-2017/((put hash here))tech}}
\paragraph{}
The value of argument regarding battery life of these two hearables should also heavily decrease in the coming years. At the end of October 2017 Samsung has announced that a considerably new generation of battery has been developed. Currently used lithium-ion batteries seem to have been pushed to it's limit  and yet it still takes a fairly long time time to charge in a fast-paced society. This problem has pushed electronics manufacturers to develop energy efficient processors. A new graphene-based battery technology should enable 45 percent more capacity and 5 times faster charging speeds.
\paragraph{}
These reasons should lead to a breakthrough in battery life factor of next year's electronics. If not at 2019, by 2020 hearing aids should receive this battery update. Combined with improvements in Bluetooth technology, these reasons should encourage both hearing aid and consumer audio manufacturers to encourage an increased number of features in coming year's hearing aids as well as PSAPs and might bring the markets closer together. 
   


