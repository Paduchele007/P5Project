\chapter{Problem Analysis}\label{ProblemAnalysis}
\section{Problem Description}
Humans have the ability to identify the source of a sound around them. In the 
field of neuroscience, this capability is called sound localization. The brain
can determine the location of a sound with very high precision, up to 2 degrees 
of space. This comes from the brain's capacity to interpret information received
from both ears.


Over the years, neuroscientists, have been trying to understand the mechanisms 
within our brains that are able to determine the location of a sound. They have 
identified two cues that are essential and sufficient for horizontal sound
localization.

Figure XX shows a circle, and a person in the middle. The person is meant to be
the listener, while the circle represents a perfectly flat plane around the listeners
head. 

In the 1790s, Giovanni Battista Venturi conducted experiments where he played a flute around 
blindfolded people and asked them to point in his direction. He concluded that the sound
amplitude difference between the two ears was the indication used for determining the direction.

Much later, Malloch proposed that the difference in time between the two ears was the sign
used for determining the direction of a sound.

Years later, scientists found neurons in the auditory center of the brain
specially adjusted for each indicator: time and intensity differences between the two ears.


For example, sound coming from 
the speaker would reach your left ear faster and be louder than the sound that 
reaches your right ear. Your brain compares these differences and tells you 
where the sound is coming from!

\todo[inline]{Same text as in introduction. Modify paragraph above, add references, add picture, write about vertical,
delimitate afterwards}

\section{Problem Delimitation}
\section{Initiating Problem}


