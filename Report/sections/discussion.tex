\chapter{Discussion}\label{ch:discussion}
This chapter seeks to understand and explain why the speech separation method worked in theory,
however failed in real life scenarios.
It also talks about future perspectives of the research done and presented in this report.
\section{Observations}
We believe, some of the factors presented below, might have had a high enough influence for the speech 
separation method to not fully function in real life scenarios.
\subsubsection{Microphones}
One of the factors that probably had the biggest impact was the quality of the microphones. As more
performant microphones were not available, we resorted on using a pair of very low price microphones
purchased online.
\subsubsection{Measurements}
Whenever recording samples, the angles of the recordings might not have been exact, leading to improper 
matching between the microphones. Additionally, even though a ruler was used to try and keep a fixed 
distance between the microphones, slight misalignment might have happened, thus giving the wrong
amount of delay samples.
\subsubsection{Software}
Software induced delay was another issue we tried to make up for as much as possible. Having used two USB 
microphones, the samples have included any latency caused by the microphones, which is different from 
recording to recording.\cite{USBLATENCY}
\newpage
\subsubsection*{Three sound sources} 
As we could get acceptable results with two sound sources we have decided to continue with 
our idea and expand on it to use it to separate one sound source from three sound sources.
Samples addition as well as shifting parts are exactly the same as what has been used with 
two sound sources. The difference came, when trying to separate one source from other two: 
We have attempted to add two shifted signals, which we shifted to the sides both of the 
sources we want to remove were placed at and then subtracting twice from this sum right 
microphone data.
Unfortunately, doing so only could separate one source in one case, we couldn't manage to 
separate other two. Overall, this method shouldn't be successful, but the fact, that we 
managed to make it work in one case is interesting. 
something} \\
The case, when separation worked we shifted signal to the side (INSERT A DELAY WITH A 
SIGN) and other signal to the side (INSERT OTHER DELAY WITH A SIGN) and then added these 
two together and removed right microphone data from this sum twice. After listening to the 
result we could hear a lot from other two sources, but the one which we were aiming to 
separate was still heard loud and clear.\\
Possible improvements: in addition to shifting, try to add opposite gain which could make 
the signals not only similar in phase but also in amplitude. This might help with the 
sound which stays in the background and is a bit audible if we try to listen to it. 
Applying this might also help with better and more consistent separation when working with 
three sound sources. \todo[inline]{maybe take out}
