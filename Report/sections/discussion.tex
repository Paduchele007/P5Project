\chapter{Discussion}\label{ch:discussion}
This chapter seeks to understand and explain why the speech separation method worked in theory,
however failed in real life scenarios.
\section{Observations}
We believe, some of the factors presented below, might have had a high enough influence for the speech 
separation method to not fully function in real life scenarios.
\subsubsection{Microphones}
One of the factors that probably had the biggest impact was the quality of the microphones. As more
performant microphones were not available, we resorted on using a pair of very low price microphones
purchased online.
\subsubsection{Measurements}
Whenever recording samples, the angles of the recordings might not have been exact, leading to improper 
matching between the microphones. Additionally, even though a ruler was used to try and keep a fixed 
distance between the microphones, slight misalignment might have happened, thus giving the wrong
amount of delay samples.
\subsubsection{Software}
Software induced delay was another issue we tried to make up for as much as possible. Having used two USB 
microphones, the samples have included any latency caused by the microphones, which is different from 
recording to recording.\cite{USBLATENCY}
\newpage
\section{Three Sound Sources} 
An attempt was made to try and separate one sound source from three sound sources. Seeing
that we were successful in trying to remove one, removing two was the next step.
We were however unable to perform this task. Not obtaining any relevant data, we choose not 
to expand on the matter.

\section{Training of the Neural Network}
The results of the Neural Network, despite the fact that they look promising and that they make a lot of progress over a short period of training, could get better and more accurate results from more training. This can be accomplished with a more powerful computer that can handle a bigger dataset and with a better GPU that can train on the respective dataset in a reasonable amount of time. 